\documentclass[12pt, a4paper]{article}

\usepackage[T2A]{fontenc}
\usepackage[utf8]{inputenc}
\usepackage[english, russian]{babel}
\usepackage[left=1.5cm, right=1.5cm, top=3.5cm, bottom=2cm]{geometry}

%============================================
\newcommand{\OGEdate}{}
\newcommand{\OGEvariant}{}
%============================================

% Настройка ссылок
\usepackage[
	unicode, pdftex, 
	colorlinks=true, 
	linkcolor=black, 
	pdfhighlight=/P
	]{hyperref} 

% 
\usepackage{fancyhdr}
\pagestyle{fancy}
\fancyhf{}
\rhead{Вариант \OGEvariant}
\lhead{ОГЭ от \OGEdate}

\usepackage{amsmath}
\usepackage{amssymb}
\usepackage{cancel}
\usepackage{hhline}
\usepackage{listings}
\usepackage{ifthen}
\usepackage{xparse}

\title{ОГЭ от \OGEdate}
\author{Виногродский Серафим}

\newcommand{\answer}[1]{
    \expandafter\newcommand\csname answer\the\value{answers} \endcsname{#1}
    
    Ответ: #1
}
\newcommand{\getanswer}[1]{
    \csname answer#1 \endcsname
}

\newcommand{\issue}[1]{
    \setcounter{answers}{#1}
    \subsection{Номер #1}
}

\renewcommand{\thesection}{}
\renewcommand{\thesubsection}{}

\begin{document}
    
    \newcounter{answers}
    \setcounter{answers}{1}
    
    % -------------------------------------
    
    \maketitle
    \label{content}
    \tableofcontents
    \thispagestyle{empty}
    \pagebreak
    
    % =====================================
    
    \section{Вариант \OGEvariant}
    
    \issue{1}
    \answer{}
    
    \issue{2}
    \answer{}
    
    \issue{3}
    \answer{}
    
    \issue{4}
    \answer{}
    
    \issue{5}
    \answer{}
    
    \issue{6}
    \answer{}
    
    \issue{7}
    \answer{}
    
    \issue{8}
    \answer{}
    
    \issue{9}
    \answer{}
    
    \issue{10}
    \answer{}
    
    \issue{11}
    \answer{}
    
    \issue{12}
    \answer{}
    
    \issue{13}
    \answer{}
    
    \issue{14}
    \answer{}
    
    \issue{15}
    \answer{}
    
    \issue{16}
    \answer{}
    
    \issue{17}
    \answer{}
    
    \issue{18}
    \answer{}
    
    \issue{19}
    В архиве находится файл с решением.
    \answer{}
    
    \pagebreak
    \subsection{Номер 20.2}
    \lstinputlisting[language=c++, basicstyle=\footnotesize, frame=TBLR]{20.2.cpp}
    
    % ========================================================
    
    \newcounter{forcounter}
    
    \NewDocumentCommand{\for}{m m O{1} m}
    {%
        \ifthenelse{#3 = 0}{}{%
            \setcounter{forcounter}{#1}%
            \ifthenelse{\value{forcounter} < #2}{%
                #4%
                \addtocounter{forcounter}{#3}%
                \for{\value{forcounter}}{#2}[#3]{#4}%
            }{}%
        }%
    } 
        
    \newcommand{\iterator}{\arabic{forcounter}}
    
    %---------------------------------------------------------

    \section{Таблица ответов}
    
    \begin{table}[h!]
        \centering
        \begin{tabular}{|*{13}{c|}}
            \hhline{|*{13}{-}|}
            \for{1}{13}{\iterator & } 13
            \\ \hhline{:*{13}{=}:}
            \for{1}{13}{\getanswer{\iterator} & } \getanswer{13}
            \\ \hhline{|*{13}{-}|}
        \end{tabular}
    \end{table}
    \begin{table}[h!]
        \centering
        \begin{tabular}{|*{6}{c|}}
            \hhline{|*{6}{-}|}
            14 & 15 & 16 & 17 & 18 & 19  \\ \hhline{*{6}{=}}
            \getanswer{14} & 
            \getanswer{15} & 
            \getanswer{16} & 
            \getanswer{17} & 
            \getanswer{18} & 
            \getanswer{19} \\ \hhline{|*{6}{-}|}
        \end{tabular}
    \end{table}
    \begin{table}[h!]
        \centering
        \begin{tabular}{|c|}
            \hline
            \hyperref[content]{К содержанию (стр. \pageref{content})} \\
            \hline  
        \end{tabular}
    \end{table}
\end{document}
